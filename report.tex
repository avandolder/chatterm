\documentclass{article}

\title{Project Report}
\author{Adam Vandolder}
\date{}

\begin{document}
	\pagenumbering{gobble}
	\maketitle
    \tableofcontents
	\newpage
	\pagenumbering{arabic}

    \section{Introduction}
    \section{Features of Application}
    \section{Tools and Techniques}
    ChatTerm relies primarily upon Python for its operation.
    More specifically, I made use of Python 3.6 and some of its libraries: sys,
    threading, typing, socket, and string, all of which come with Python by
    default. I also made use of Python's curses library, which only comes with
    Python on Linux and Mac OS X, so on Windows the `windows-curses` library
    must be installed as well. When writing the project, I mainly used Vim.
    ChatTerm is based around a server-client architecture, using the TCP
    protocol via the socket library to provide communication between the
    processes. The server is multi-threaded, spinning off a new thread to handle
    each new client that connects. The client is single-threaded, and uses
    curses in order to maintain a Text-User-Interface (TUI) that shows both the
    messages from the server as well as the users input at the same time.
    \section{Implementation Details}
    \section{Conclusion}
    \section{References}
\end{document}
