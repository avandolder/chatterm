\documentclass{article}

\usepackage{hyperref}

\title{Project Report}
\author{Adam Vandolder}
\date{}

\begin{document}
    \pagenumbering{gobble}
    \maketitle
    \tableofcontents
    \newpage
    \pagenumbering{arabic}
    \linespread{1.6}
    \selectfont

    \section{Introduction}
    For my Computer Networks project, I decided to create a client-server chat
    application, called termchat, in Python. termchat is so-called because both
    the server and client are entirely terminal-based applications, utilizing a
    text-based user interface made with Curses. It uses the TCP protocol in
    order to communicate between the clients and server.

    \section{Features of Application}
    termchat is a fully functional terminal-based chat server and client, with
    commands inspired by the IRC protocol. The client uses a text-based user
    interface, while the server isn't interactive. Many clients can connect to a
    given server by its domain and port number, and can then communicate with
    each other. Servers provide a series of different channels that clients can
    join in order to chat about more specific topics, and clients can make new
    channels on-the-fly. Clients are free to set non-conflicting nick names for
    themselves, and request lists of current users connected to the server.
    Clients can also direct message any other client connected to the same
    server, regardless of which channels those clients are currently on. The
    client's text window can be cleared of current messages as well.
    
    \section{Tools and Techniques}
    termchat relies primarily upon Python for its operation.
    More specifically, I made use of Python 3.6 and some of its libraries: sys,
    threading, typing, socket, and string, all of which come with Python by
    default. I also made use of Python's curses library, which only comes with
    Python on Linux and Mac OS X, so on Windows the `windows-curses' library
    must be installed as well. When writing the project, I mainly used Vim. In
    order to create the report, I used LaTex and the texlive package. In order
    to generate the Windows binaries, I used the PyInstaller program. I used git
    for source control, and termchat's repository is available at
    \url{https://github.com/avandolder/termchat/}. The mypy utility was used to
    statically type check the server and client programs.

    \section{Implementation Details}
    termchat is based around a client-server architecture, using the TCP
    protocol via the socket library to provide communication between the
    processes. The server is multi-threaded, spinning off a new thread to handle
    each new client that connects. The client is single-threaded, and uses
    curses in order to maintain a Text-User-Interface (TUI) that shows both the
    messages from the server as well as the users input at the same time. The
    git repository contains the list of commits, showing the progress made over
    time.

    I started the project by creating basic client and servers programs that
    could connect via TCP, based off the examples provided in the Python
    documentation. I initially planned to use the socketserver library in order
    to ease the creation of my server program, however, I soon realized it would
    be awkward to get it to allow communication between the various clients, so
    I instead opted for a custom server using just the socket and threading
    libraries. Next, I added a basic TUI with curses into the client, so I could
    show both the incoming messages from the server as well as the users input
    at the same time, without needing to bring in a full GUI. I then added a
    series of basic commands, such as /nick, /clear, etc. while also adding
    additional input features, like backspacing and input history.

    termchat makes use of Python's optional type annotations that were added in
    Python 3.6. Using the mypy utility and threading module, you can now
    statically type check your Python programs, removing an entire class of type
    errors that previously could only be diagnosed at runtime.

    \section{Execution Instructions} 
    If you are going to be running termchat on Windows, I have included
    pre-built binaries for the server and client programs. These are accessible
    as the dist/server/server.exe and dist/client/client.exe respectively. You
    will still need to pass the domain and port number as command-line arguments
    to the server program as shown below.

    In order to run termchat, you will need Python installed, and at it will
    need to be at least version 3.6. In order to run the client, you will also
    need the Curses library installed. If you are running Python on Linux or
    Mac, then Curses will be installed by default. However, on Windows, you will
    need to install the \emph{windows-curses} library. This can be installed
    with the following command:\\
    \textbf{python -m pip install windows-curses}\\
    Then, in order to launch the server, you will have to run the following
    command from the project directory:\\
    \textbf{python src/server.py domain portnumber}\\
    If you want to run a server instance that interacts with clients on the same
    machine, set \emph{domain} to localhost and \emph{portnumber} to a high
    number that is likely unused, such as 9999.\\
    In order to launch a client, run the following command:\\
    \textbf{python src/client.py}\\
    You can then enter the \emph{/help} command in order to get a list of the
    supported commands. Enter \emph{/server} followed by the domain and port
    number you gave the server instance in order to join your local server.

    \section{Conclusion}
    In the process of creating this project, I learned a substantial amount
    about how to work with TCP sockets as well as how servers work, especially
    in regards to handling many clients at once with multi-threading. All in
    all, this project was a successful learning experience.

    \section{References}
    When making this project I relied heavily upon the main Python
    documentation, especially:\\
    sockets docs: \url{https://docs.python.org/3.6/library/socket.html}\\
    threading docs: \url{https://docs.python.org/3.6/library/threading.html}\\
    curses docs: \url{https://docs.python.org/3.6/library/curses.html}\\
    Curses Programming with Python:
    \url{https://docs.python.org/3.6/howto/curses.html}\\
    I also utilized Curses Programming in Python on Dev Dungeon:
    \url{https://www.devdungeon.com/content/curses-programming-python}\\
\end{document}
